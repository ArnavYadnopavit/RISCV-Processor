\documentclass[12pt,a4paper]{article}
\usepackage[a4paper,margin=1in]{geometry}
\usepackage{setspace}
\usepackage{titlesec}
\usepackage{graphicx}
\usepackage{booktabs}

\setstretch{1.2}
\titleformat{\section}{\bfseries\large}{\thesection.}{0.5em}{}
\titleformat{\subsection}{\bfseries}{\thesubsection}{0.5em}{}

\begin{document}

\begin{titlepage}
    \centering
    {\Large \textbf{CS2323: Computer Architecture}}\\[0.5cm]
    {\large \textbf{Project and Specification Proposal}}\\[2.5cm]
    
    {\LARGE \textbf{Design and FPGA Implementation of a 64-bit RISC-V Processor with FPU and UART Interface}}\\[2cm]

    \textbf{Submitted By:}\\[0.5cm]
    \begin{tabular}{rl}
        Krishna H. Patil & (EE24BTECH11036)\\
        Arnav Yadnopavit & (EE24BTECH11007)
    \end{tabular}\\[1.5cm]
    
    \vfill
    Indian Institute of Technology Hyderabad\\
    \vspace{0.5cm}
    \today
\end{titlepage}

\section{Project Scope}

The objective of this project is to design and demonstrate a 64-bit RISC-V processor on an FPGA board (Arty A7 or Pynq-Z2). The design will evolve in multiple stages—beginning with a single-cycle implementation and progressively extending to a multicycle version with additional components such as a Floating-Point Unit (FPU) and UART interface for communication with a host computer.

\subsection*{Key Features Planned}
\begin{itemize}
    \item Implementation of the RV64I base instruction set (integer operations).
    \item Support for arithmetic, logical, load/store, and branch instructions.
    \item Development of a single-cycle CPU initially, followed by a multicycle version for improved resource utilization.
    \item Addition of a Floating-Point Unit (FPU) supporting IEEE-754 operations.
    \item Integration of a UART interface for serial communication and demonstration of results.
\end{itemize}

The final goal is to demonstrate the working of the 64-bit RISC-V processor executing sample programs (e.g., factorial, matrix operations, floating-point arithmetic) and verify outputs via UART on a connected terminal.

\subsection*{Minimum Deliverable}
The minimum deliverable planned for the project is a \textbf{64-bit multicycle RISC-V CPU} capable of correctly executing the core RV64I instruction set. This includes arithmetic, logical, load/store, and control-flow instructions, verified both through simulation and on FPGA hardware.

\section{Implementation Plan}

\subsection*{Phase 1 – Single Cycle RISC-V Processor (RV64I)}
Design datapath components: Register File, ALU, Control Unit, Immediate Generator, Program Counter, and Instruction/Data Memory. Implement instruction fetch, decode, execute, memory access, and write-back in a single clock cycle. Verify correct execution using simulation for sample programs.

\subsection*{Phase 2 – Multicycle RISC-V Processor}
Modify datapath to reuse functional units across multiple cycles. Implement a finite state machine (FSM) for control, reducing hardware duplication and improving timing closure. Validate correctness using the same instruction tests.

\subsection*{Phase 3 – Floating-Point Unit (FPU) Integration}
Implement or integrate a basic IEEE-754 compliant single-precision FPU supporting add, subtract, and multiply operations. Extend control logic to recognize floating-point instructions. Test with mixed integer-floating point programs.

\subsection*{Phase 4 – UART Integration}
Add UART transmit and receive modules. Implement a memory-mapped I/O interface for UART communication. Demonstrate program output via a PC terminal.(some interface to work with the CPU in a more comfy way)

\section{Verification Plan}

\begin{center}
\begin{tabular}{@{}lll@{}}
\toprule
\textbf{Stage} & \textbf{Verification Method} & \textbf{Tools Used} \\ \midrule
Single Cycle & Simulation of instruction sequences (add, branch, load/store) & ModelSim / Vivado Simulator \\
Multicycle & FSM state trace verification, timing analysis & Vivado \\
FPU & Testbench for FP operations vs. software reference (Python / GCC) & ModelSim / Python script \\
UART & Loopback and terminal tests for input/output & Tera Term / PuTTY \\
Full System & Run compiled RISC-V assembly programs via \texttt{\$readmemh} & Vivado / FPGA board \\
\bottomrule
\end{tabular}
\end{center}

\section{Tools and Resources}

\begin{itemize}
    \item \textbf{Hardware Description Language:} Verilog
    \item \textbf{Simulation:} Vivado Simulator / ModelSim
    \item \textbf{Synthesis \& Implementation:} Xilinx Vivado
    \item \textbf{FPGA Boards:} Arty A7 / Pynq-Z2
    \item \textbf{Software Toolchain:} RISC-V GCC / \texttt{objcopy} for .hex generation
    \item \textbf{Optional Testing:} Python reference model for numerical verification
\end{itemize}

\section{Expected Outcome}

\begin{itemize}
    \item Working 64-bit RISC-V CPU running on FPGA.
    \item UART output showing computation results (e.g., factorial, floating-point arithmetic).
    \item Demonstration video or live hardware run during evaluation.
\end{itemize}

\end{document}

